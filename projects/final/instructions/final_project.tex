\documentclass{article}

\usepackage{fullpage}
\usepackage{hyperref}
\hypersetup{
  colorlinks = true
}
\usepackage{listings}
\lstset{basicstyle=\ttfamily\footnotesize,breaklines=true}


\title{{\sc final project and presentation}}
\date{3/22/2020}
\begin{document}
\maketitle
\section{Summary}
This document outlines the final project which will have at least the artifact an {\em individual} presentation.
We begin with some general expectation and time boxing in \ref{sec:import-dates} {\bf General Expectations and Important Dates}.
A final project can go a number of directions and possibilities are discussed in \ref{sec:proj-ideas} {\bf Project Ideas and ingredients}.

\section{General Expectation and Important Dates} \label{sec:import-dates}
\begin{enumerate}
\item {\em  You should choose a project that interests you and explores some aspect of deep reinforcement learning.
    This could be any of the following:}
  \begin{itemize}
  \item Present a paper or a collection of papers on an idea
  \item Train an agent to perform a task we have not done and present details of how to do it.
  \item Explore training methods or other software considerations that allow for faster/more efficient training.
  \end{itemize}

\item {\em You may work in a group, but there should be enough material to make individual presentations. This might be done with:}
  \begin{itemize}
  \item Presenting a number of related works on a topic (e.g. centralized training decentralized execution, soft actor-critc, etc.)
  \item Training an number of agents to do related tasks for comparison (e.g. roboschool, etc. comes to mind)
  \end{itemize}
  
\item {\em The following table proposes some dates to keep things on track, these will be adjusted as needed.}

\begin{centering}
  \begin{tabular}{c|c}
    Date & What \\
    \hline
    27 March 2020 & Fill out the following \href{form}{} \\
    \hline
    1 April 2020 & Write a summary/abstract for your project presentation and work plan \\
    \hline
    3 April 2020 & Feedback from instructor \\
    \hline
    8 April 2020 & Progress report on work plan (instructor feedback as necessary) \\
    \hline
    15 April 2020 & Progress report on work plan (instructor feedback as necessary) \\
    \hline
    15 April 2020 & Progress report on work plan (instructor feedback as necessary) \\
    \hline
    22 April 2020 & Draft of slides or other artifacts \\
    \hline
    24 April 2020 & Final instructor feedback
  \end{tabular}
\end{centering}
\end{enumerate}

\section{Project Ideas and Ingredients} \label{sec:proj-ideas}
The follow give some example projects. This list should be much longer, but it's a start.
\begin{itemize}
\item Compare training with with APPO and APEX vs. PPO and DQN
\item Train an agent for Mujoco or roboschool task
\item Present the evolution strategies approach to deep RL
\item Train an agent to play othello
\item Train an agent to play go
\item Train an agent for vizdoom
\item Present work on distributional RL
\item Present work on model based learning
\item Present work on max entropy RL
\item Present work monte carlo tree search
\item Train Atari games in minutes using evolution strategies
\item Develop a new environment for a practical application (e.g. smart meters)
\end{itemize}

\end{document}
